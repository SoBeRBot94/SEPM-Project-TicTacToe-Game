% Generated by Sphinx.
\def\sphinxdocclass{report}
\newif\ifsphinxKeepOldNames \sphinxKeepOldNamestrue
\documentclass[letterpaper,10pt,english]{sphinxmanual}
\usepackage{iftex}

\ifPDFTeX
  \usepackage[utf8]{inputenc}
\fi
\ifdefined\DeclareUnicodeCharacter
  \DeclareUnicodeCharacter{00A0}{\nobreakspace}
\fi
\usepackage{cmap}
\usepackage[T1]{fontenc}
\usepackage{amsmath,amssymb,amstext}
\usepackage{babel}
\usepackage{times}
\usepackage[Bjarne]{fncychap}
\usepackage{longtable}
\usepackage{sphinx}
\usepackage{multirow}
\usepackage{eqparbox}


\addto\captionsenglish{\renewcommand{\figurename}{Fig.\@ }}
\addto\captionsenglish{\renewcommand{\tablename}{Table }}
\SetupFloatingEnvironment{literal-block}{name=Listing }

\addto\extrasenglish{\def\pageautorefname{page}}

\setcounter{tocdepth}{1}


\title{TicTacToe-GE Documentation}
\date{Oct 07, 2017}
\release{0.1}
\author{Group K}
\newcommand{\sphinxlogo}{}
\renewcommand{\releasename}{Release}
\makeindex

\makeatletter
\def\PYG@reset{\let\PYG@it=\relax \let\PYG@bf=\relax%
    \let\PYG@ul=\relax \let\PYG@tc=\relax%
    \let\PYG@bc=\relax \let\PYG@ff=\relax}
\def\PYG@tok#1{\csname PYG@tok@#1\endcsname}
\def\PYG@toks#1+{\ifx\relax#1\empty\else%
    \PYG@tok{#1}\expandafter\PYG@toks\fi}
\def\PYG@do#1{\PYG@bc{\PYG@tc{\PYG@ul{%
    \PYG@it{\PYG@bf{\PYG@ff{#1}}}}}}}
\def\PYG#1#2{\PYG@reset\PYG@toks#1+\relax+\PYG@do{#2}}

\expandafter\def\csname PYG@tok@gd\endcsname{\def\PYG@tc##1{\textcolor[rgb]{0.63,0.00,0.00}{##1}}}
\expandafter\def\csname PYG@tok@gu\endcsname{\let\PYG@bf=\textbf\def\PYG@tc##1{\textcolor[rgb]{0.50,0.00,0.50}{##1}}}
\expandafter\def\csname PYG@tok@gt\endcsname{\def\PYG@tc##1{\textcolor[rgb]{0.00,0.27,0.87}{##1}}}
\expandafter\def\csname PYG@tok@gs\endcsname{\let\PYG@bf=\textbf}
\expandafter\def\csname PYG@tok@gr\endcsname{\def\PYG@tc##1{\textcolor[rgb]{1.00,0.00,0.00}{##1}}}
\expandafter\def\csname PYG@tok@cm\endcsname{\let\PYG@it=\textit\def\PYG@tc##1{\textcolor[rgb]{0.25,0.50,0.56}{##1}}}
\expandafter\def\csname PYG@tok@vg\endcsname{\def\PYG@tc##1{\textcolor[rgb]{0.73,0.38,0.84}{##1}}}
\expandafter\def\csname PYG@tok@vi\endcsname{\def\PYG@tc##1{\textcolor[rgb]{0.73,0.38,0.84}{##1}}}
\expandafter\def\csname PYG@tok@mh\endcsname{\def\PYG@tc##1{\textcolor[rgb]{0.13,0.50,0.31}{##1}}}
\expandafter\def\csname PYG@tok@cs\endcsname{\def\PYG@tc##1{\textcolor[rgb]{0.25,0.50,0.56}{##1}}\def\PYG@bc##1{\setlength{\fboxsep}{0pt}\colorbox[rgb]{1.00,0.94,0.94}{\strut ##1}}}
\expandafter\def\csname PYG@tok@ge\endcsname{\let\PYG@it=\textit}
\expandafter\def\csname PYG@tok@vc\endcsname{\def\PYG@tc##1{\textcolor[rgb]{0.73,0.38,0.84}{##1}}}
\expandafter\def\csname PYG@tok@il\endcsname{\def\PYG@tc##1{\textcolor[rgb]{0.13,0.50,0.31}{##1}}}
\expandafter\def\csname PYG@tok@go\endcsname{\def\PYG@tc##1{\textcolor[rgb]{0.20,0.20,0.20}{##1}}}
\expandafter\def\csname PYG@tok@cp\endcsname{\def\PYG@tc##1{\textcolor[rgb]{0.00,0.44,0.13}{##1}}}
\expandafter\def\csname PYG@tok@gi\endcsname{\def\PYG@tc##1{\textcolor[rgb]{0.00,0.63,0.00}{##1}}}
\expandafter\def\csname PYG@tok@gh\endcsname{\let\PYG@bf=\textbf\def\PYG@tc##1{\textcolor[rgb]{0.00,0.00,0.50}{##1}}}
\expandafter\def\csname PYG@tok@ni\endcsname{\let\PYG@bf=\textbf\def\PYG@tc##1{\textcolor[rgb]{0.84,0.33,0.22}{##1}}}
\expandafter\def\csname PYG@tok@nl\endcsname{\let\PYG@bf=\textbf\def\PYG@tc##1{\textcolor[rgb]{0.00,0.13,0.44}{##1}}}
\expandafter\def\csname PYG@tok@nn\endcsname{\let\PYG@bf=\textbf\def\PYG@tc##1{\textcolor[rgb]{0.05,0.52,0.71}{##1}}}
\expandafter\def\csname PYG@tok@no\endcsname{\def\PYG@tc##1{\textcolor[rgb]{0.38,0.68,0.84}{##1}}}
\expandafter\def\csname PYG@tok@na\endcsname{\def\PYG@tc##1{\textcolor[rgb]{0.25,0.44,0.63}{##1}}}
\expandafter\def\csname PYG@tok@nb\endcsname{\def\PYG@tc##1{\textcolor[rgb]{0.00,0.44,0.13}{##1}}}
\expandafter\def\csname PYG@tok@nc\endcsname{\let\PYG@bf=\textbf\def\PYG@tc##1{\textcolor[rgb]{0.05,0.52,0.71}{##1}}}
\expandafter\def\csname PYG@tok@nd\endcsname{\let\PYG@bf=\textbf\def\PYG@tc##1{\textcolor[rgb]{0.33,0.33,0.33}{##1}}}
\expandafter\def\csname PYG@tok@ne\endcsname{\def\PYG@tc##1{\textcolor[rgb]{0.00,0.44,0.13}{##1}}}
\expandafter\def\csname PYG@tok@nf\endcsname{\def\PYG@tc##1{\textcolor[rgb]{0.02,0.16,0.49}{##1}}}
\expandafter\def\csname PYG@tok@si\endcsname{\let\PYG@it=\textit\def\PYG@tc##1{\textcolor[rgb]{0.44,0.63,0.82}{##1}}}
\expandafter\def\csname PYG@tok@s2\endcsname{\def\PYG@tc##1{\textcolor[rgb]{0.25,0.44,0.63}{##1}}}
\expandafter\def\csname PYG@tok@nt\endcsname{\let\PYG@bf=\textbf\def\PYG@tc##1{\textcolor[rgb]{0.02,0.16,0.45}{##1}}}
\expandafter\def\csname PYG@tok@nv\endcsname{\def\PYG@tc##1{\textcolor[rgb]{0.73,0.38,0.84}{##1}}}
\expandafter\def\csname PYG@tok@s1\endcsname{\def\PYG@tc##1{\textcolor[rgb]{0.25,0.44,0.63}{##1}}}
\expandafter\def\csname PYG@tok@ch\endcsname{\let\PYG@it=\textit\def\PYG@tc##1{\textcolor[rgb]{0.25,0.50,0.56}{##1}}}
\expandafter\def\csname PYG@tok@m\endcsname{\def\PYG@tc##1{\textcolor[rgb]{0.13,0.50,0.31}{##1}}}
\expandafter\def\csname PYG@tok@gp\endcsname{\let\PYG@bf=\textbf\def\PYG@tc##1{\textcolor[rgb]{0.78,0.36,0.04}{##1}}}
\expandafter\def\csname PYG@tok@sh\endcsname{\def\PYG@tc##1{\textcolor[rgb]{0.25,0.44,0.63}{##1}}}
\expandafter\def\csname PYG@tok@ow\endcsname{\let\PYG@bf=\textbf\def\PYG@tc##1{\textcolor[rgb]{0.00,0.44,0.13}{##1}}}
\expandafter\def\csname PYG@tok@sx\endcsname{\def\PYG@tc##1{\textcolor[rgb]{0.78,0.36,0.04}{##1}}}
\expandafter\def\csname PYG@tok@bp\endcsname{\def\PYG@tc##1{\textcolor[rgb]{0.00,0.44,0.13}{##1}}}
\expandafter\def\csname PYG@tok@c1\endcsname{\let\PYG@it=\textit\def\PYG@tc##1{\textcolor[rgb]{0.25,0.50,0.56}{##1}}}
\expandafter\def\csname PYG@tok@o\endcsname{\def\PYG@tc##1{\textcolor[rgb]{0.40,0.40,0.40}{##1}}}
\expandafter\def\csname PYG@tok@kc\endcsname{\let\PYG@bf=\textbf\def\PYG@tc##1{\textcolor[rgb]{0.00,0.44,0.13}{##1}}}
\expandafter\def\csname PYG@tok@c\endcsname{\let\PYG@it=\textit\def\PYG@tc##1{\textcolor[rgb]{0.25,0.50,0.56}{##1}}}
\expandafter\def\csname PYG@tok@mf\endcsname{\def\PYG@tc##1{\textcolor[rgb]{0.13,0.50,0.31}{##1}}}
\expandafter\def\csname PYG@tok@err\endcsname{\def\PYG@bc##1{\setlength{\fboxsep}{0pt}\fcolorbox[rgb]{1.00,0.00,0.00}{1,1,1}{\strut ##1}}}
\expandafter\def\csname PYG@tok@mb\endcsname{\def\PYG@tc##1{\textcolor[rgb]{0.13,0.50,0.31}{##1}}}
\expandafter\def\csname PYG@tok@ss\endcsname{\def\PYG@tc##1{\textcolor[rgb]{0.32,0.47,0.09}{##1}}}
\expandafter\def\csname PYG@tok@sr\endcsname{\def\PYG@tc##1{\textcolor[rgb]{0.14,0.33,0.53}{##1}}}
\expandafter\def\csname PYG@tok@mo\endcsname{\def\PYG@tc##1{\textcolor[rgb]{0.13,0.50,0.31}{##1}}}
\expandafter\def\csname PYG@tok@kd\endcsname{\let\PYG@bf=\textbf\def\PYG@tc##1{\textcolor[rgb]{0.00,0.44,0.13}{##1}}}
\expandafter\def\csname PYG@tok@mi\endcsname{\def\PYG@tc##1{\textcolor[rgb]{0.13,0.50,0.31}{##1}}}
\expandafter\def\csname PYG@tok@kn\endcsname{\let\PYG@bf=\textbf\def\PYG@tc##1{\textcolor[rgb]{0.00,0.44,0.13}{##1}}}
\expandafter\def\csname PYG@tok@cpf\endcsname{\let\PYG@it=\textit\def\PYG@tc##1{\textcolor[rgb]{0.25,0.50,0.56}{##1}}}
\expandafter\def\csname PYG@tok@kr\endcsname{\let\PYG@bf=\textbf\def\PYG@tc##1{\textcolor[rgb]{0.00,0.44,0.13}{##1}}}
\expandafter\def\csname PYG@tok@s\endcsname{\def\PYG@tc##1{\textcolor[rgb]{0.25,0.44,0.63}{##1}}}
\expandafter\def\csname PYG@tok@kp\endcsname{\def\PYG@tc##1{\textcolor[rgb]{0.00,0.44,0.13}{##1}}}
\expandafter\def\csname PYG@tok@w\endcsname{\def\PYG@tc##1{\textcolor[rgb]{0.73,0.73,0.73}{##1}}}
\expandafter\def\csname PYG@tok@kt\endcsname{\def\PYG@tc##1{\textcolor[rgb]{0.56,0.13,0.00}{##1}}}
\expandafter\def\csname PYG@tok@sc\endcsname{\def\PYG@tc##1{\textcolor[rgb]{0.25,0.44,0.63}{##1}}}
\expandafter\def\csname PYG@tok@sb\endcsname{\def\PYG@tc##1{\textcolor[rgb]{0.25,0.44,0.63}{##1}}}
\expandafter\def\csname PYG@tok@k\endcsname{\let\PYG@bf=\textbf\def\PYG@tc##1{\textcolor[rgb]{0.00,0.44,0.13}{##1}}}
\expandafter\def\csname PYG@tok@se\endcsname{\let\PYG@bf=\textbf\def\PYG@tc##1{\textcolor[rgb]{0.25,0.44,0.63}{##1}}}
\expandafter\def\csname PYG@tok@sd\endcsname{\let\PYG@it=\textit\def\PYG@tc##1{\textcolor[rgb]{0.25,0.44,0.63}{##1}}}

\def\PYGZbs{\char`\\}
\def\PYGZus{\char`\_}
\def\PYGZob{\char`\{}
\def\PYGZcb{\char`\}}
\def\PYGZca{\char`\^}
\def\PYGZam{\char`\&}
\def\PYGZlt{\char`\<}
\def\PYGZgt{\char`\>}
\def\PYGZsh{\char`\#}
\def\PYGZpc{\char`\%}
\def\PYGZdl{\char`\$}
\def\PYGZhy{\char`\-}
\def\PYGZsq{\char`\'}
\def\PYGZdq{\char`\"}
\def\PYGZti{\char`\~}
% for compatibility with earlier versions
\def\PYGZat{@}
\def\PYGZlb{[}
\def\PYGZrb{]}
\makeatother

\renewcommand\PYGZsq{\textquotesingle}

\begin{document}

\maketitle
\tableofcontents
\phantomsection\label{index::doc}


Contents:
\phantomsection\label{index:module-GameEngine}\index{GameEngine (module)}\index{GameEngine (class in GameEngine)}

\begin{fulllineitems}
\phantomsection\label{index:GameEngine.GameEngine}\pysigline{\sphinxstrong{class }\sphinxcode{GameEngine.}\sphinxbfcode{GameEngine}}
When initializing GameEngine, a board 3x3 board is filled with the values of None. The first player
is always player `X', whilst the second player is always the player `O'.
\index{changePlayer() (GameEngine.GameEngine method)}

\begin{fulllineitems}
\phantomsection\label{index:GameEngine.GameEngine.changePlayer}\pysiglinewithargsret{\sphinxbfcode{changePlayer}}{}{}
Changes the current player that is to play on the board. Saves it to self.

\end{fulllineitems}

\index{getBoard() (GameEngine.GameEngine method)}

\begin{fulllineitems}
\phantomsection\label{index:GameEngine.GameEngine.getBoard}\pysiglinewithargsret{\sphinxbfcode{getBoard}}{}{}~\begin{quote}\begin{description}
\item[{Returns}] \leavevmode
The current board.

\item[{Return type}] \leavevmode
List{[}List{[}int{]}{]}

\end{description}\end{quote}

\end{fulllineitems}

\index{getPlayer() (GameEngine.GameEngine method)}

\begin{fulllineitems}
\phantomsection\label{index:GameEngine.GameEngine.getPlayer}\pysiglinewithargsret{\sphinxbfcode{getPlayer}}{}{}~\begin{quote}\begin{description}
\item[{Returns}] \leavevmode
The player whose turn it is - `X' or `O'.

\item[{Return type}] \leavevmode
String

\end{description}\end{quote}

\end{fulllineitems}

\index{getResult() (GameEngine.GameEngine method)}

\begin{fulllineitems}
\phantomsection\label{index:GameEngine.GameEngine.getResult}\pysiglinewithargsret{\sphinxbfcode{getResult}}{\emph{board}}{}~\begin{quote}\begin{description}
\item[{Parameters}] \leavevmode
\textbf{\texttt{board}} (\emph{\texttt{List{[}List{[}int{]}{]}}}) -- The 3x3 board from GameEngine.

\item[{Returns}] \leavevmode
The result from the board with the values being either `X', `O', or tie

\item[{Return type}] \leavevmode
String

\end{description}\end{quote}

\end{fulllineitems}

\index{isFinished() (GameEngine.GameEngine method)}

\begin{fulllineitems}
\phantomsection\label{index:GameEngine.GameEngine.isFinished}\pysiglinewithargsret{\sphinxbfcode{isFinished}}{\emph{board}}{}~\begin{quote}\begin{description}
\item[{Parameters}] \leavevmode
\textbf{\texttt{board}} (\emph{\texttt{List{[}List{[}int{]}{]}}}) -- The 3x3 board from GameEngine.

\item[{Returns}] \leavevmode
A boolean stating if the game is finished or if there are still moves left to be placed.

\item[{Return type}] \leavevmode
Boolean

\end{description}\end{quote}

\end{fulllineitems}

\index{resetBoard() (GameEngine.GameEngine method)}

\begin{fulllineitems}
\phantomsection\label{index:GameEngine.GameEngine.resetBoard}\pysiglinewithargsret{\sphinxbfcode{resetBoard}}{}{}
Resets the board for a new game.

\end{fulllineitems}

\index{updateBoard() (GameEngine.GameEngine method)}

\begin{fulllineitems}
\phantomsection\label{index:GameEngine.GameEngine.updateBoard}\pysiglinewithargsret{\sphinxbfcode{updateBoard}}{\emph{player}, \emph{move}}{}~\begin{quote}\begin{description}
\item[{Parameters}] \leavevmode\begin{itemize}
\item {} 
\textbf{\texttt{player}} (\emph{\texttt{String}}) -- The player that is making a move.

\item {} 
\textbf{\texttt{move}} (\emph{\texttt{tuple}}) -- The move that the player has chosen to play.

\end{itemize}

\item[{Returns}] \leavevmode
Nothing. The updated board is saved in self.board.

\end{description}\end{quote}

\begin{notice}{note}{Note:}
Raises a ValueError if the move is invalid.
\end{notice}

\end{fulllineitems}


\end{fulllineitems}

\phantomsection\label{index:module-Player}\index{Player (module)}\index{Player (class in Player)}

\begin{fulllineitems}
\phantomsection\label{index:Player.Player}\pysiglinewithargsret{\sphinxstrong{class }\sphinxcode{Player.}\sphinxbfcode{Player}}{\emph{typePlayer}, \emph{difficulty='default'}}{}
Bases: \sphinxcode{object}

The class which controls if it's a human player or an AI.
Can take two parameters as input, where the type of the player (AI or user) is mandatory
A second parameter, difficulty, is not needed when initializing a user type of player
but it is mandatory when initializing a AI. If trying to initialize an AI without the difficulty
parameter, a ValueError is raised from the \_AI class. If trying to initialize the Player class
without specifying that the player is a `ai' or a `user', a ValueError is raised notifying the client of this.
The only public facing method which is used by the client is nextMove.
\index{nextMove() (Player.Player method)}

\begin{fulllineitems}
\phantomsection\label{index:Player.Player.nextMove}\pysiglinewithargsret{\sphinxbfcode{nextMove}}{\emph{board}, \emph{currentPlayer}}{}
Runs the method nextMove for the class which was initialized.
\begin{quote}\begin{description}
\item[{Parameters}] \leavevmode\begin{itemize}
\item {} 
\textbf{\texttt{board}} -- The 3x3 board from GameEngine

\item {} 
\textbf{\texttt{currentPlayer}} -- The player who is making the next move (X or O)

\end{itemize}

\item[{Returns}] \leavevmode
tuple with the row and column for the next move. On the form of (rowIdx, colIdx)

\end{description}\end{quote}

\end{fulllineitems}


\end{fulllineitems}

\index{\_User (class in Player)}

\begin{fulllineitems}
\phantomsection\label{index:Player._User}\pysigline{\sphinxstrong{class }\sphinxcode{Player.}\sphinxbfcode{\_User}}
Class which is used by the human player. Shouldn't be called directly, instead, call it via the class Player.
\index{getMove() (Player.\_User method)}

\begin{fulllineitems}
\phantomsection\label{index:Player._User.getMove}\pysiglinewithargsret{\sphinxbfcode{getMove}}{}{}
Returns the value stored in self.move which is the next chosen move for the AI.
\begin{quote}\begin{description}
\item[{Returns}] \leavevmode
The with the row and column for the next move. On the form of (rowIdx, colIdx)

\item[{Return type}] \leavevmode
tuple

\end{description}\end{quote}

\end{fulllineitems}

\index{getPossibleMoves() (Player.\_User method)}

\begin{fulllineitems}
\phantomsection\label{index:Player._User.getPossibleMoves}\pysiglinewithargsret{\sphinxbfcode{getPossibleMoves}}{\emph{board}}{}~\begin{quote}\begin{description}
\item[{Parameters}] \leavevmode
\textbf{\texttt{board}} (\emph{\texttt{List{[}List{[}int{]}{]}}}) -- The 3x3 board from GameEngine.

\item[{Returns}] \leavevmode
The possible moves left on the board.

\item[{Return type}] \leavevmode
List{[}tuple{]}

\end{description}\end{quote}

\end{fulllineitems}

\index{nextMove() (Player.\_User method)}

\begin{fulllineitems}
\phantomsection\label{index:Player._User.nextMove}\pysiglinewithargsret{\sphinxbfcode{nextMove}}{\emph{board}, \emph{currentPlayer}}{}
Runs the method nextMove for the class which was initialized.
\begin{quote}\begin{description}
\item[{Parameters}] \leavevmode\begin{itemize}
\item {} 
\textbf{\texttt{board}} -- The 3x3 board from GameEngine. On the form of List{[}List,List,List{]}

\item {} 
\textbf{\texttt{currentPlayer}} -- The player who is making the next move (X or O)

\end{itemize}

\item[{Returns}] \leavevmode
The row and column for the next move. On the form of (rowIdx, colIdx)

\item[{Return type}] \leavevmode
tuple

\end{description}\end{quote}

\end{fulllineitems}


\end{fulllineitems}

\phantomsection\label{index:module-AI}\index{AI (module)}\index{\_AI (class in AI)}

\begin{fulllineitems}
\phantomsection\label{index:AI._AI}\pysiglinewithargsret{\sphinxstrong{class }\sphinxcode{AI.}\sphinxbfcode{\_AI}}{\emph{difficulty}}{}
Bases: \sphinxcode{object}

Class which is used to initialize an AI type of player. Shouldn't be called directly, instead,
call it via the class Player. The class \_AI, which receives the difficulty parameter from the initialization
of the Player class, uses it to be able to initialize the correct difficulty of the AI.
If the difficulty parameter isn't one of the following: hard, medium or easy, a ValueError is raised
notifying the client of this.
\index{nextMove() (AI.\_AI method)}

\begin{fulllineitems}
\phantomsection\label{index:AI._AI.nextMove}\pysiglinewithargsret{\sphinxbfcode{nextMove}}{\emph{board}, \emph{currentPlayer}}{}
Runs the method nextMove for the class which was initialized.
\begin{quote}\begin{description}
\item[{Parameters}] \leavevmode\begin{itemize}
\item {} 
\textbf{\texttt{board}} -- The 3x3 board from GameEngine. On the form of List{[}List,List,List{]}

\item {} 
\textbf{\texttt{currentPlayer}} -- The player who is making the next move (X or O)

\end{itemize}

\item[{Returns}] \leavevmode
The row and column for the next move. On the form of (rowIdx, colIdx)

\item[{Return type}] \leavevmode
tuple

\end{description}\end{quote}

\end{fulllineitems}


\end{fulllineitems}

\index{\_AIEasy (class in AI)}

\begin{fulllineitems}
\phantomsection\label{index:AI._AIEasy}\pysigline{\sphinxstrong{class }\sphinxcode{AI.}\sphinxbfcode{\_AIEasy}}~\index{getMove() (AI.\_AIEasy method)}

\begin{fulllineitems}
\phantomsection\label{index:AI._AIEasy.getMove}\pysiglinewithargsret{\sphinxbfcode{getMove}}{}{}
Returns the value stored in self.move which is the next chosen move for the AI.
\begin{quote}\begin{description}
\item[{Returns}] \leavevmode
The with the row and column for the next move. On the form of (rowIdx, colIdx)

\item[{Return type}] \leavevmode
tuple

\end{description}\end{quote}

\end{fulllineitems}

\index{getPossibleMoves() (AI.\_AIEasy method)}

\begin{fulllineitems}
\phantomsection\label{index:AI._AIEasy.getPossibleMoves}\pysiglinewithargsret{\sphinxbfcode{getPossibleMoves}}{\emph{board}}{}~\begin{quote}\begin{description}
\item[{Parameters}] \leavevmode
\textbf{\texttt{board}} (\emph{\texttt{List{[}List{[}int{]}{]}}}) -- The 3x3 board from GameEngine.

\item[{Returns}] \leavevmode
The possible moves left on the board.

\item[{Return type}] \leavevmode
List{[}tuple{]}

\end{description}\end{quote}

\end{fulllineitems}

\index{nextMove() (AI.\_AIEasy method)}

\begin{fulllineitems}
\phantomsection\label{index:AI._AIEasy.nextMove}\pysiglinewithargsret{\sphinxbfcode{nextMove}}{\emph{board}, \emph{currentPlayer}}{}
Finds the next move for the AI.
\begin{quote}\begin{description}
\item[{Parameters}] \leavevmode\begin{itemize}
\item {} 
\textbf{\texttt{board}} (\emph{\texttt{List{[}List{[}int{]}{]}}}) -- The 3x3 board from GameEngine.

\item {} 
\textbf{\texttt{currentPlayer}} (\emph{\texttt{String}}) -- The player who is making the next move (`X' or `O')

\end{itemize}

\end{description}\end{quote}

\end{fulllineitems}


\end{fulllineitems}

\index{\_AIMedium (class in AI)}

\begin{fulllineitems}
\phantomsection\label{index:AI._AIMedium}\pysigline{\sphinxstrong{class }\sphinxcode{AI.}\sphinxbfcode{\_AIMedium}}~\index{getMove() (AI.\_AIMedium method)}

\begin{fulllineitems}
\phantomsection\label{index:AI._AIMedium.getMove}\pysiglinewithargsret{\sphinxbfcode{getMove}}{}{}
Returns the value stored in self.move which is the next chosen move for the AI.
\begin{quote}\begin{description}
\item[{Returns}] \leavevmode
The with the row and column for the next move. On the form of (rowIdx, colIdx)

\item[{Return type}] \leavevmode
tuple

\end{description}\end{quote}

\end{fulllineitems}

\index{getPossibleMoves() (AI.\_AIMedium method)}

\begin{fulllineitems}
\phantomsection\label{index:AI._AIMedium.getPossibleMoves}\pysiglinewithargsret{\sphinxbfcode{getPossibleMoves}}{\emph{board}}{}~\begin{quote}\begin{description}
\item[{Parameters}] \leavevmode
\textbf{\texttt{board}} (\emph{\texttt{List{[}List{[}int{]}{]}}}) -- The 3x3 board from GameEngine.

\item[{Returns}] \leavevmode
The possible moves left on the board.

\item[{Return type}] \leavevmode
List{[}tuple{]}

\end{description}\end{quote}

\end{fulllineitems}

\index{nextMove() (AI.\_AIMedium method)}

\begin{fulllineitems}
\phantomsection\label{index:AI._AIMedium.nextMove}\pysiglinewithargsret{\sphinxbfcode{nextMove}}{\emph{board}, \emph{currentPlayer}}{}
Finds the next move for the AI.
\begin{quote}\begin{description}
\item[{Parameters}] \leavevmode\begin{itemize}
\item {} 
\textbf{\texttt{board}} (\emph{\texttt{List{[}List{[}int{]}{]}}}) -- The 3x3 board from GameEngine.

\item {} 
\textbf{\texttt{currentPlayer}} (\emph{\texttt{String}}) -- The player who is making the next move (`X' or `O')

\end{itemize}

\end{description}\end{quote}

\end{fulllineitems}


\end{fulllineitems}

\index{\_AIHard (class in AI)}

\begin{fulllineitems}
\phantomsection\label{index:AI._AIHard}\pysigline{\sphinxstrong{class }\sphinxcode{AI.}\sphinxbfcode{\_AIHard}}
Class which is the hardest AI. Shouldn't be called directly, instead, call it via the class \_AI which is
called via the class Player. It has no initialization parameters as it is given that it should go via the \_AI class.
\index{getMove() (AI.\_AIHard method)}

\begin{fulllineitems}
\phantomsection\label{index:AI._AIHard.getMove}\pysiglinewithargsret{\sphinxbfcode{getMove}}{}{}
Returns the value stored in self.move which is the next chosen move for the AI.
\begin{quote}\begin{description}
\item[{Returns}] \leavevmode
The with the row and column for the next move. On the form of (rowIdx, colIdx)

\item[{Return type}] \leavevmode
tuple

\end{description}\end{quote}

\end{fulllineitems}

\index{getPossibleMoves() (AI.\_AIHard method)}

\begin{fulllineitems}
\phantomsection\label{index:AI._AIHard.getPossibleMoves}\pysiglinewithargsret{\sphinxbfcode{getPossibleMoves}}{\emph{board}}{}~\begin{quote}\begin{description}
\item[{Parameters}] \leavevmode
\textbf{\texttt{board}} (\emph{\texttt{List{[}List{[}int{]}{]}}}) -- The 3x3 board from GameEngine.

\item[{Returns}] \leavevmode
The possible moves left on the board.

\item[{Return type}] \leavevmode
List{[}tuple{]}

\end{description}\end{quote}

\end{fulllineitems}

\index{isTerminalState() (AI.\_AIHard method)}

\begin{fulllineitems}
\phantomsection\label{index:AI._AIHard.isTerminalState}\pysiglinewithargsret{\sphinxbfcode{isTerminalState}}{\emph{board}}{}~\begin{quote}\begin{description}
\item[{Parameters}] \leavevmode
\textbf{\texttt{board}} (\emph{\texttt{List{[}List{[}int{]}{]}}}) -- The 3x3 board from GameEngine.

\item[{Returns}] \leavevmode
The state of the current board.

\end{description}\end{quote}

\end{fulllineitems}

\index{nextMove() (AI.\_AIHard method)}

\begin{fulllineitems}
\phantomsection\label{index:AI._AIHard.nextMove}\pysiglinewithargsret{\sphinxbfcode{nextMove}}{\emph{board}, \emph{currentPlayer}}{}
Finds the next move for the AI.
\begin{quote}\begin{description}
\item[{Parameters}] \leavevmode\begin{itemize}
\item {} 
\textbf{\texttt{board}} (\emph{\texttt{List{[}List{[}int{]}{]}}}) -- The 3x3 board from GameEngine.

\item {} 
\textbf{\texttt{currentPlayer}} (\emph{\texttt{String}}) -- The player who is making the next move (`X' or `O')

\end{itemize}

\end{description}\end{quote}

\end{fulllineitems}

\index{score() (AI.\_AIHard method)}

\begin{fulllineitems}
\phantomsection\label{index:AI._AIHard.score}\pysiglinewithargsret{\sphinxbfcode{score}}{\emph{board}}{}
The score to be retrieved for the minimax algorithm.
\begin{quote}\begin{description}
\item[{Parameters}] \leavevmode
\textbf{\texttt{board}} (\emph{\texttt{List{[}List{[}int{]}{]}}}) -- The 3x3 board from GameEngine.

\item[{Returns}] \leavevmode
The score used for the minimax algorithm

\item[{Return type}] \leavevmode
int

\end{description}\end{quote}

\end{fulllineitems}


\end{fulllineitems}



\chapter{Indices and tables}
\label{index:welcome-to-tictactoe-ge-s-documentation}\label{index:indices-and-tables}\begin{itemize}
\item {} 
\DUrole{xref,std,std-ref}{genindex}

\item {} 
\DUrole{xref,std,std-ref}{modindex}

\item {} 
\DUrole{xref,std,std-ref}{search}

\end{itemize}


\renewcommand{\indexname}{Python Module Index}
\begin{theindex}
\def\bigletter#1{{\Large\sffamily#1}\nopagebreak\vspace{1mm}}
\bigletter{a}
\item {\texttt{AI}}, \pageref{index:module-AI}
\indexspace
\bigletter{g}
\item {\texttt{GameEngine}}, \pageref{index:module-GameEngine}
\indexspace
\bigletter{p}
\item {\texttt{Player}}, \pageref{index:module-Player}
\end{theindex}

\renewcommand{\indexname}{Index}
\printindex
\end{document}
